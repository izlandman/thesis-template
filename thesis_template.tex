\documentclass[reqno,12pt,oneside]{report} %right-side equation numbering, 12 point font, print one-sided
%\documentclass[reqno,12pt,twoside,openright]{report} %right-side equation numbering, 12 point font, print two-sided, Chapters start on odd pages. Rackham only accepts one-sided, so this is for personal printings.
\usepackage{algorithm}
\usepackage[noend]{algorithmic}
\usepackage{fancyhdr}
\usepackage{verbatim}
\usepackage{mathrsfs}
\usepackage{url}
\usepackage[cmex10]{amsmath}

% this is style file that is being edited to support adapation toward a Temple style disseration
\usepackage{rac}
% these are specified by Temple, as of Spring 2017 the requirements were
% 1.5 inch left margin
% 1 inch right margin
% 1 inch top margin
% 1 inch bottom margin
\usepackage[top=1in, bottom=1in, left=1.5in, right=1in, includefoot]{geometry}
%\usepackage{geometry}
\renewcommand{\theequation}{\thesection-\arabic{equation}}
\usepackage{array}
% correct bad hyphenation here

%\usepackage{rac1}         % Use Rackham thesis style file
%\usepackage{aas_macros}
\usepackage{apacite}
\usepackage{apacdoc}
% To allow the reading of ADS journal references in the bibliography
%\usepackage[intlimits]{amsmath} % Puts the limits of integrals on top and bottom
\usepackage{amsxtra}     % Use various AMS packages
\usepackage{amsthm}
\usepackage{amssymb}
\usepackage{amsfonts}
\usepackage{graphicx}    % Add some packages for figures. Read epslatex.pdf on ctan.tug.org
\usepackage{rotating}
\usepackage{color}
\usepackage{epsfig}
\usepackage{subfigure}   % To make subfigures. Read subfigure.pdf on ctan.tug.org
\usepackage{verbatim}
\usepackage{natbib}      % Allows you to use BibTeX
\usepackage[printonlyused]{acronym} % For the List of Abbreviations. Read acronym.pdf on ctan.tug.org
\usepackage{setspace}    % Allows you to specify the line spacing
\doublespacing           % \onehalfspacing for 1.5 spacing, \doublespacing for 2.0 spacing.
\newcommand{\sun}{\ensuremath{\odot}} % sun symbol is \sun

%%%%%%%%%%%%%%%%%%%%%%%%%%%%%%%%%%%%%%%%%%%%%%%%%%%%%%%%%%%%%%%%%%%%%%%%%%%%%%%

% Various theorem environments. All of the following have the same numbering
% system as theorem.

\theoremstyle{plain}
\newtheorem{theorem}{Theorem}
\newtheorem{prop}[theorem]{Proposition}
\newtheorem{corollary}[theorem]{Corollary}
\newtheorem{lemma}[theorem]{Lemma}
\newtheorem{question}[theorem]{Question}
\newtheorem{conjecture}[theorem]{Conjecture}
\newtheorem{assumption}[theorem]{Assumption}

\theoremstyle{definition}
\newtheorem{definition}[theorem]{Definition}
\newtheorem{notation}[theorem]{Notation}
\newtheorem{condition}[theorem]{Condition}
\newtheorem{example}[theorem]{Example}
\newtheorem{introduction}[theorem]{Introduction}

\theoremstyle{remark}
\newtheorem{remark}[theorem]{Remark}
%%%%%%%%%%%%%%%%%%%%%%%%%%%%%%%%%%%%%%%%%%%%%%%%%%%%%%%%%%%%%%%%%%%%%%%%%%%%%%%

\numberwithin{theorem}{chapter}
% Numbers theorems "x.y" where x is the section number, y is the theorem number

%\renewcommand{\thetheorem}{\arabic{chapter}.\arabic{theorem}}

%\makeatletter
%\let\c@equation\c@theorem
%\makeatother
% This sequence of commands will incorporate equation numbering into the theorem numbering scheme

%\renewcommand{\theenumi}{(\roman{enumi})}

%This command creates a box marked ``To Do'' around text.
%To use type \todo{  insert text here  }.

\newcommand{\todo}[1]{\vspace{5 mm}\par \noindent
\marginpar{\textsc{To Do}}
\framebox{\begin{minipage}[c]{0.95 \textwidth}
\tt\begin{center} #1 \end{center}\end{minipage}}\vspace{5 mm}\par}

%%%%%%%%%%%%%%%%%%%%%%%%%%%%%%%%%%%%%%%%%%%%%%%%%%%%%%%%%%%%%%%%%%%%%%%%%%%%%%%
\begin{document}
% Title page as required by Temple University dissertation guidelines
\titlepage{Snappy Title Goes Here}{the Temple University Graduate Board}{DOCTOR OF PHILOSOPHY}
{Christian Radcliffe Ward}{May / August / December, 20XX}
{Dr. Your Advisor , Advisor, Dept. of Electrial and Computer Engineering \\
 Dr. Member One, Dept. of Z and X \\
 Dr. Member Two, Dept. of Z and X \\
 Dr. Member Three, Dept. of Z and X \\
 Dr. Member Four, External Reader, Dept. of Z and X \\
}
% Begin the front matter as required by Rackham dissertation guidelines
\initializefrontsections

% Optional Frontispiece, don't believe Temple supports this feature
%  \frontispiece{\includegraphics[width=4in]{xiali.PNG} a cool picture here.}

% Optional, but recommended, Copyright page. if included must be labeled as page number ii
\copyrightpage{Copyright}{Xiali Hei}

% Optional in-dissertation Abstract Page
\startabstractpage
\lipsum[1-3]
\label{Abstract}

% Optional Acknowledgements page
\startacknowledgementspage

This book could not have been written without Dr. Xiaojiang Du, who
encouraged and challenged me through my academic program. He never
accepted less than my best efforts. Thank you. What is written in
this book are materials that I found in my papers. A
special thanks to the authors mentioned in the bibliography page. I
would like to acknowledge and extend my heartfelt gratitude to another advisor
of mine --Dr. Shan Lin. Most especially to my family,
friends and my son, Peiheng Ni. Words alone cannot express what I
owe them for their encouragement and whose patient love enabled me
to complete this book. A special thanks to Jie Wu for comments on my
editing. The book was developed from ideas originally published in the Globecom 2010, Infocom 2011 and %\cite{hanselman}.
As always it was editor Xuemin (Sherman) Shen who provided the shelter conditions under which the work could take place: thanks to him
for this and many other things.

\label{Acknowledgements}

% Optional Dedication page
\dedicationpage{Words.}

%Optional Preface page
%\startprefacepage
%\input{Preface}
%\label{Preface}

% Table of contents, list of figures, etc.
\tableofcontents     % Required
\listoffigures       % Required if there is more than one figure
\listoftables        % Required if there is more than one table
%\listofmaps          % Required if there is more than one map
%\listofappendices    % Required if there is more than one appendix
%\listofabbreviations % Optional. Abbreviations should be stored in a file named abbr.tex

\startthechapters
% The individual files for each of the chapters are put here.
% Save each chapter of your thesis to a separate tex file
% and then use the \input command to include this file in your
% thesis.  For instance you can save a file to "intro.tex" and
% then type \input{intro}.
\chapter{INTRODUCTION}
 \label{chap:Particles}
 
Securing IMDs is a very challenging task due
to their very limiting resource constraints in terms of energy
supply, processing power, storage space, etc. An IMD is implanted in a patient's body and is expected to operate
for several months or years. Typical IMDs are powered by a non-rechargeable battery, and replacement of the battery requires surgery. Re-charging an IMD via an external RF electromagnetic source causes thermal effects in body tissues and thus is not recommended. Unlike general medical sensors that may use AA-type or renewable (e.g., solar) batteries, an IMD typically uses silver vanadium oxide batteries and therefore is very vulnerable to the Resource Depletion (RD) attacks
\cite{Globecom}. The RD attacks include a number of attacks that try
to consume as much energy as possible, such as Denial of Service
(DoS) attacks and forced authentication attacks (discussed later).
These kinds of attacks can be easily launched but are difficult to
defend against. A number of literatures \cite{Wang},
\cite{RFIDsecurity}, \cite{DoS}, \cite{DoS1}, \cite{DHPC} have
studied DoS attacks on wireless sensor networks. Raymond and Midkiff
\cite{DoS} provide a survey of DoS attacks against sensor networks.
However, the security schemes designed for sensor networks cannot be
directly applied to IMDs, because IMDs have much less available resources than typical sensor nodes. For example, a Mica2 mote sensor has 128KB programmable memory and 512K data memory \cite{mote}, while an IMD may have less than 10KB memory. Furthermore, it is much easier to replace the battery for a sensor node than for an IMD. Hence, special light-weight security schemes need to be designed for IMDs. Another
difference between sensor nodes and IMDs is that an IMD is implanted in a patient's body and directly involves a human (the patient). Hence, effective security schemes for IMDs may utilize the human in their design.

During emergencies, a patient (say Bob) may be unconscious and
cannot provide his credentials (such as a token or a key) to the
medical personnel, nor can he show his ID or inform medical
personnel about his medical information. In addition, neither
device-based schemes nor family-based schemes \cite{Fang} can be
used if the patient has an emergency outside his home country. In
this case, the safety of patients outweighs the security and privacy
concerns of IMDs. A good access control scheme should satisfy security, privacy and safety requirements.

%In this book, we present related works in Chapter 2, which consists
%of defense solutions for IMDs in normal situation as well as during emergencies. In Chapter 3, we discuss the Resource Depletion (RD) attacks and a defense scheme based on a patient's IMD access pattern. In Chapter 3, we present a light-weight biometrics based secure access control scheme for IMDs during emergencies. The conclusion and our future directions are given in Chapter 5.

%\input{referenc1}

\chapter{LITERATURE REVIEW}
 \label{chap:Particles}
 \section{Start Here}

\subsection{More Here}

\subsection{And Again}

\section{Restart!}


\chapter{A UNIFIED SECURE FRAMEWORK FOR WIRELESS MEDICAL DEVICES USING PATIENT'S CELL PHONE}
 \label{chap:Particles}
 \section{Start Here}

\subsection{More Here}

\subsection{And Again}

\section{Restart!}


\chapter{NEAR FIELD COMMUNICATION BASED ACCESS CONTROL FOR WIRELESS MEDICAL DEVICES}
 \label{chap:Particles}
 \section{Start Here}

\subsection{More Here}

\subsection{And Again}

\section{Restart!}


\chapter{A PATIENT ACCESS PATTERN BASED ACCESS CONTROL SCHEME}
 \label{chap:Particles}
 \section{Start Here}

\subsection{More Here}

\subsection{And Again}

\section{Restart!}


\chapter{PATIENT INFUSION PATTERN BASED ACCESS CONTROL SCHEMES FOR WIRELESS INSULIN PUMP SYSTEM}
 \label{chap:Particles}
 \section{Start Here}

\subsection{More Here}

\subsection{And Again}

\section{Restart!}


\chapter{BIOMETRICS BASED TWO-LEVEL SECURE ACCESS CONTROL FOR IMPLANTABLE MEDICAL DEVICES DURING EMERGENCIES}
 \label{chap:Particles}
 \section{Start Here}

\subsection{More Here}

\subsection{And Again}

\section{Restart!}


\chapter{CONCLUSION}
 \label{chap:Particles}
 \section{Start Here}

\subsection{More Here}

\subsection{And Again}

\section{Restart!}



\startbibliography
 \begin{singlespace} % Bibliography must be single spaced
  \bibliographystyle{apacite}
  %\bibliographystyle{agu04}
  %\bibliographystyle{aas_macros}
  \bibliography{referencelist1}
   %\input{Ref/referenc1}
  %\bibliography{References}   %Use the BibTeX file ``References.bib''.
 \end{singlespace}

 \startappendices
 \appendix{Appendix A}
 \label{Carelink USB Driver Decoding}
 \section{Start Here}

\subsection{More Here}

\subsection{And Again}

\section{Restart!}


\end{document}
