
Securing IMDs is a very challenging task due
to their very limiting resource constraints in terms of energy
supply, processing power, storage space, etc. An IMD is implanted in a patient's body and is expected to operate
for several months or years. Typical IMDs are powered by a non-rechargeable battery, and replacement of the battery requires surgery. Re-charging an IMD via an external RF electromagnetic source causes thermal effects in body tissues and thus is not recommended. Unlike general medical sensors that may use AA-type or renewable (e.g., solar) batteries, an IMD typically uses silver vanadium oxide batteries and therefore is very vulnerable to the Resource Depletion (RD) attacks
\cite{Globecom}. The RD attacks include a number of attacks that try
to consume as much energy as possible, such as Denial of Service
(DoS) attacks and forced authentication attacks (discussed later).
These kinds of attacks can be easily launched but are difficult to
defend against. A number of literatures \cite{Wang},
\cite{RFIDsecurity}, \cite{DoS}, \cite{DoS1}, \cite{DHPC} have
studied DoS attacks on wireless sensor networks. Raymond and Midkiff
\cite{DoS} provide a survey of DoS attacks against sensor networks.
However, the security schemes designed for sensor networks cannot be
directly applied to IMDs, because IMDs have much less available resources than typical sensor nodes. For example, a Mica2 mote sensor has 128KB programmable memory and 512K data memory \cite{mote}, while an IMD may have less than 10KB memory. Furthermore, it is much easier to replace the battery for a sensor node than for an IMD. Hence, special light-weight security schemes need to be designed for IMDs. Another
difference between sensor nodes and IMDs is that an IMD is implanted in a patient's body and directly involves a human (the patient). Hence, effective security schemes for IMDs may utilize the human in their design.

During emergencies, a patient (say Bob) may be unconscious and
cannot provide his credentials (such as a token or a key) to the
medical personnel, nor can he show his ID or inform medical
personnel about his medical information. In addition, neither
device-based schemes nor family-based schemes \cite{Fang} can be
used if the patient has an emergency outside his home country. In
this case, the safety of patients outweighs the security and privacy
concerns of IMDs. A good access control scheme should satisfy security, privacy and safety requirements.

%In this book, we present related works in Chapter 2, which consists
%of defense solutions for IMDs in normal situation as well as during emergencies. In Chapter 3, we discuss the Resource Depletion (RD) attacks and a defense scheme based on a patient's IMD access pattern. In Chapter 3, we present a light-weight biometrics based secure access control scheme for IMDs during emergencies. The conclusion and our future directions are given in Chapter 5.

%\input{referenc1}